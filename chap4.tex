%%% -*-LaTeX-*-

\chapter{SweetPea Language}

SweetPea achieves its goal of creating unbiased experimental sequences from high-level experimental designs by relying on a SAT sampler. Writing experimental designs in the language of a SAT sampler-- that is, as a boolean formula-- is highly uncomfortable. SweetPea, therefore, conviniently bridges the gap between scientists and SAT samplers by creating a wormhole: scientists describe their experimental designs in the terms they usually use in their domain, and SweetPea translates that high-level specification into an enormous low-level boolean formula. The SweetPea runtime then passes the SAT sampler Unigen that formula, and translates the low-level assignments into a high-level experimental sequence which satisfies the design. That experimental sequence is guaranteed to be approximately as likely as any other sequences which fulfills the design, reassuring scientists that they did not accidentally introduce bias because of the way the sequence was constructed.

In this chapter, we'll document the features of the SweetPea language. They largely follow the components we saw in the Overview chapter-- experimental designs consist of trials described in terms of factors and levels, and constraints and relationships over those trials, described here in terms of windows and derivation functions, and counting and balancing constraints. Finally, the full experiment is described in terms of a design, a crossing and experimental blocks. Here we'll look at a variety of experiments which motivate these various features, and the high-level language we provide to describe those experiments. In the next chapter we'll take a close look at how these features are efficiently encoded into a boolean formula.


\section{Components of an Experiment}

We briefly saw the parts of an experiment with the Stroop test in Chapter 2. Now let's look at the range of experiments we wish to model, and how our representations of these experiments are amenable to SAT.

\subsection{Descriptions of Stimuli}

TODO diagrams

Each component of a stimuli is represented as a \emph{factor} which can take on a disrete number of \emph{levels}. In the Stroop experiment this was represented as the ink color being red of blue, and the text color being red or blue.

Experiments typically have 2 to 7 factors with 2 to 4 levels each, resulting in roughly 50 to 200 stimuli. Often the ideal goal is to \emph{fully cross} all of the factors to produce all possible combinations of stimuli, but sometimes factors are left out of the full crossing. For example, a subject can only reliably be tested for so long before they are tired, so sometimes factors which are deemed to be irrelevant to the control variables and variables of interest are excluded from the full crossing.

Levels may be nested into sub-levels. One reason to want to do this is to apply different constraints to each of the sublevels. For instance, a color factor may have nested sublevels of light colors (like yellow and pink) and dark colors (like navy and maroon).

Sometimes researchers may want to define factors with levels sampled from a continuous distribution, for example to model a spectrum of light or sound. Sweetpea currently does not support this other than by the research manually discretizing the continoum into several discrete values. This case is particularly challenging for sweetpea because it is difficult to phrase sampling a continious distribution as a boolean formula; for more thought on how we could better support this see Future Work.

\subsection{Ordering Constraints}

Researchers must have some control over the ordering of stimuli to be able to test their hypothesis. For example, if you are testing whether the presence of a certain level effects the subject's reaction time, you need control over when that level occurs and how often it occurs in relation to other levels. More generally, you need to be able to constrain the presence or absense of relationships which you define over the levels. Since levels can be grouped into arbitrary sub-levels, you should more generally be able to specify constraints on relationships over the sub-levels.

TODO: about diagrams

One example of such a relationship is a \emph{transition} constraint, which specifies whether each level is followed by the same level (a repetition) or a different level (a switch). A common type of transition constraint is the specification that a certain level shouldn't appear more than k-times in a row.

Another example is a \emph{congruence} constraint, which specifies whether the levels of different factors are, according to a user-specified metric, complementary or conflicting. A congruence constraint for the Stroop is whether the color of the ink and the text are the same (the word "red" in red ink) or not.

Another common relationship is a \emph{balancing} constraint. A pair of levels is balanced if each instance shows up the same number of times; a full crossing of all levels is a fully balanced experiment.

Constraints can also be defined in terms of other constraints. For instance, in the Stroop test one may want to balance the number of congruent and incongruent levels, or to balance the transitions between congruent and incongruent levels.

More generally, new experiments may need to define relationships between levels that other existing experiments don't have. SweetPea must have enough flexibility to allow researchers to define these new relationships.

\subsection{Experimental Design and Balancing}

Experiments which include every possible combination of levels are \emph{fully crossed} designs. Fully crossings are desirable because they allow the researcher to test the full space of possible stimuli.

Sometimes, however, an experiment is either too large or too small. An experiment can be too large if it has too many elements in the full crossing because there are limits on how many tasks a subject can perform in a single session reliably. If an experiment is too large it can either be restated as a full crossing of only a subset of the factors (with the non-crossed factors being assigned at random, with uniform probability), or it can be divided across subjects or across experimental sessions. An experiment can be too small if the full crossing has too few stimuli to reliably draw conclusions. In this case, the experiment can either be multiple full crossings back-to-back, or the contents of multiple full crossings combined into a larger experimental block.

Fully-crossing a design is a form of balancing-- it is the specification that each level occurs as frequently as each other level and that each level occurs once.

Sometimes the constraint of using a full-crossing makes other types of balancing impossible because the levels are over-constrained. Consider, for instance, the Stroop test with 3 different colors instead of two: then it isn't possible to both fully-cross the design and to balance the congruent and incongruent levels. (This is because there are 3 congruent levels, and 6 incongruent levels). One possible desired outcome in this case is for SweetPea to report that an experiment is over-constrained.

Another alternative is that if balancing the congruent and incongruent levels is more important than a fully-crossed design, then we can instead create a \emph{weighted crossing} instead. In a weighted crossing we can specify that we wish to under-sample certain levels (the incongruent ones in our example) or over-sample other levels (the congruent ones). Weighted crossings also allow us to express the experiments mentioned above, where the full crossing has either too many or too few stimuli to be practical.

\subsection{Experimental Structure}

An experiment is composed of one or more experimental blocks; an experimental block is a sequence of stimuli. Experiments may be organized into blocks for several reasons: there may be prologue or epilogue blocks which set up some condition for the subject, the experiment may be designed to be run in several related sessions, or consist of several sub-experiments.

It is impossible to balance some commonly desired constraints with a single block; for instance, it is not possible to have as many "repeat" transitions as "switch" transitions in the fully-balanced Stroop experiment because there are an even number of stimuli, resulting in an odd number of transitions. This means it's not possible to have as many of one transition as the other.

There are two approaches to resolving this issue. The first is nearly-balancing the levels: instead of requiring that there are exactly the same number of each level, require that there are the same number plus-or-minus one. The other solution is to create a "throw-away" block with a single stimuli in it which is used to satisfy the transition balancing constraint without being included in the full-crossing constraint.

As mentioned in the previous section, it is also sometimes useful to have multiple blocks to repeat stimuli for an experiment with too few stimuli or to break up an experiment with too many stimuli across several sessions or subjects.

\section{SweetPea Primitives}

Now that we've seen a range of experiments that we wish to represent, let's see how we can represent them in SweetPea. Here we'll look at the high-level encodings we use, and why those encodings are amenable to being represented in SAT. In the next chapter we'll look at the SAT encodings in more detail.

\subsection{Factors and Levels}
Factors have names, and a possibly nested list of levels, each of which also has a name. Names are used to specify the experimental sequence that the runtime eventually returns, but they do not need to be strings. They can be any "printable" data-type; this can be useful because names are referenced in user-defined constraints as we'll see in the next subsection.

\begin{verbatim}
example_factor = ("example_name", ["example_level"])
boolean_factor = ("booleans", [False, True])
text      = ("text",      ["red", "blue"])
light_colors = ("light_colors", ["pink", "aqua"])
dark_colors = ("dark_colors", ["navy", "maroon"])
ink_color = ("ink_color", [light_colors, dark_colors])
\end{verbatim}

In the resulting experimental sequence, exactly one choice of level is selected per trial. Internally, the experimental sequences is represented as [TODO DIAGRAM] a list of trials, where each trial is represented as a list all possible levels. There are multiple possible encodings, but the one we chose here is a "one-hot" encoding, where each level is a boolean variable which indicates whether or not it is chosen. All the boolean variables corresponding to levels of a single factor can then have an additional constraint applied that exactly one must be true. This has the effect of "chosing" one level for every factor in the trial, thereby specifying the trial. Because we do this for every trial, a satisfying assignment to these constraints represents a valid experimental sequence. See [diagram] for an example, and see Chater 6 for a more detailed discussion on the trade-offs of the choice of this, and other, encodings.

Nested levels are handled in exactly the same way- the nesting is just a way to reference a grouping in the high-level representation. This is useful because it can be used to talk about relationships between groups, ie, in constraints. At the boolean representation level, however, these groupings are irrelevant and the structure is flattened.

\subsection{Deriving Levels with Windows}

TODO COUNTING CONSTRAINTS

How do we represent relationships between levels and enforce constraints over them? All of the constraints we are considering relate to order, so let's consider a symbolic window which slides over the list of trials. A window has a width and a stride. The width specifies how many trials to consider at once; the stride specifies how many trials to move forward by when the window moves. When combined with user-defined functions, windows allow the user to specify new "derived" levels and factors based on the values of the levels within the window. See the diagram [DIAGRAM].

Consider the transition relationship, which for a given factor specifies whether each following trial has the same value for that factor (a repetition) or a different value (a switch). We will represent this relationship as a new, derived, factor. This factor has two levels: one level which indicates that the following stimuli was a repetition, the other that it was a switch. What are the parameters for the window for this relationship? Because we consider two trials at a time, and want to consider all pairs of consecutive trials, the width is two, and the stride is one.

Given this conceptual window, the user can then provide a function for each level of the derived factor which specifies, based on the values of the levels which are "in scope" in the window, whether the derived level is selected or not. For the transition example, this may look like:

\begin{verbatim}
def repetition(ink_colors):
  if ink_colors[0] == ink_colors[1]:
    return True
  return False
\end{verbatim}

The repetition function takes a list of ink colors. Because we're considering all consecutive pairs of trials, the window size is two, so the length of the ink colors list is also two. Conceptually, this function will be applied to all consecutive pairs of trials and report whether those trials are repeat or not.

We can use this repetition function to define a new pair of levels. Recall that levels act as indicators; when a level of a trial is true, that means that trial possesses the quality that level represents. The level derived using a transition-wide window and the repetition function defined above is true when the trial is has the same ink color property as the trial before it. Because levels can be defined to be any [printable] type, derivation functions can use any property of the level name as part of their logic.

Putting it all together, these derived levels are represented in SweetPea as:

\begin{verbatim}
ink_color = ("ink_color", ["red", "blue"])
text      = ("text",      ["red", "blue"])

def repetition(ink_colors):
  if ink_colors[0] == ink_colors[1]:
    return True
  return False

repeat_level  = DerivedLevel("rep", Transition(repetition, [ink_color, ink_color]))
switch_level  = DerivedLevel("swi", Transition(repetition, [ink_color, ink_color], False))
transitionFactor = Factor("congruent?", [repeat_level, switch_level])

\end{verbatim}

Let's consider another use-case. In the Stroop experiment, it is useful to ask whether or not a trial is \emph{congruent}, meaning whether the ink color and the text specify the same color or not. We can define this relationship by considering a WithinTrial window, in which the width is one (since we consider a single trial at a time), and the stride is also one. The derivation function for congruence is similiar to the one for repetitions; here however we always compare the first (and only) item in the ink colors and texts lists.

\begin{verbatim}
ink_color = ("ink_color", ["red", "blue"])
text      = ("text",      ["red", "blue"])

def congruent(ink_colors, texts):
  if ink_colors[0] == text[0]:
    return True
  return False

con_level  = DerivedLevel("rep", WithinTrial(congruent, [ink_color, ink_color]))
inc_level  = DerivedLevel("swi", WithinTrial(congruent, [ink_color, ink_color], False))
congruenceFactor = Factor("congruent?", [con_level, inc_level])

\end{verbatim}

Windows allow us to represent, more generally than just transitions and within-trial constraints, relationships across a sequence of trials. Some experiments may wish to define relationships which skip trials in the middle of the sequences. An example is an experiment with a "bait" trial: subjects are asked to respond based on  TODO clarify. Additionally, changing the stride allows for things like specifying a constraint over the first half, and then second half of the experiment.


- why is this amenable to SAT? we've got discrete boolean elements whose state depends only upon the state of other boolean elements.

- two options: you can either define a factor with a partition (levels are "where the func is true" and "where the func is false") or by defining and combining levels


\subsection{Experimental Design, Balancing and Experimental Structure}

Once we've described the trials in terms of levels and factors, we need to specify the set of trials that make up the experiment. The \emph{design} of the experiment is the specification of factors represent a trial. To specify the set of trials in the experiment we need to specify the \emph{crossing} of the design.

A common crossing which SweetPea supports is a \emph{full crossing}. A full crossing consists of exactly one instance of every possible trial. For the small Stroop experiment this means four trials, with one instance each of: (red ink, red text), (red ink, blue text), (blue ink, red text), (blue ink, blue text). Another supported crossing is a variation of a full crossing which is each instance appears exactly $k$ times instead of exactly once. A reason to exclude a factor from a design is to exclude it from the full crossing.

In a full crossing, each of the levels is \emph{balanced} with respect to all of the other levels; they all appear equally frequently. An experiment may additionally specify other levels which must be balanced, including derived levels. For example, we may wish to balance the congruent and incongruent levels.

The final aspect we need to specify is the structure of the experimental sequence. An experimental sequence is a list of one or more experimental blocks. Each experiemental block consists of a crossing of the experimental design. Some experiments may wish to specify multiple experimental blocks to represent prologue or epilogue blocks, or to split an experiment across multiple sessions.

Having discussed and motivated all of the aspects of the SweetPea language, in the next chapter we'll see how these elements are encoded as a boolean formula.

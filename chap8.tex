\chapter{Future Work}

- chapter summary

%%% ~~~~~~~~~~~~~~~~~~~~~~~~~~~~~~~~~~~~~~~~~~~~~~~~~~~~~~~~~~~~~~~~~~~~~~~~~~~
\section{Beyond Psychology}

- other science domains and what would need to change

%%% ~~~~~~~~~~~~~~~~~~~~~~~~~~~~~~~~~~~~~~~~~~~~~~~~~~~~~~~~~~~~~~~~~~~~~~~~~~~
\section{Future Language}

- section summary: discussed wishes in chapter 2 section 1, these are the ones that are currently unimplemented

\subsection{Weighted Crossings}

- discussed weighted crossings as a necessary component of experiments in chapter 2; not yet implemented

\subsection{Sampling Continuous Factors}

- this is challenging because how does this get translated to SAT?

\subsection{Automated Experimental Design}

- ANOVA experimental design

- need to make sure that this is correct in the domain in many different flavors of experiment

\subsection{Syntactic Sugar}

- don't have to write the name of the factor when the level name is unique

-

%%% ~~~~~~~~~~~~~~~~~~~~~~~~~~~~~~~~~~~~~~~~~~~~~~~~~~~~~~~~~~~~~~~~~~~~~~~~~~~
\section{Future Runtime}

- section summary

\subsection{Verified Core}

- the motivation for SweetPea is that it will make it easy to write correct experiments; that only holds if sweetpea doesn't itself have bugs.

- the tsietin transform is ripe for bugs because it's doing a non-human-readable transformation

- would be cool to formally verify that the transformations are correct


\subsection{Debugging unSAT experiments}

- why is this a problem: usability: if it's unSAT it just say unSAT but not why. Not very useful.

- can get "minimally unsat core" from SAT solver, maybe can translate that back to user-defined levels to guess what the problem is

\subsection{Iterative experimental design and Partial Satisfiability}

- really want to know if experiment is over-constrained

- maybe could try specify subsets to try to iterately find the most constraints that can be simultaneously satisfied -- solvers let you push / pop clauses

\subsection{Optimizations}

- xor constraints

- truth table simplification (Quine–McCluskey)

- choice of SAT encodings and variable v. clauses

%%% -*-LaTeX-*-


\chapter{Related Work}

SweetPea is a computation tool for pyschologists; it is a solver-aided language which translates between the vocabulary which psychologists employ and a SAT formula. To the best of our knowledge, SweetPea is the only system to address this specific problem, however there is a large body of work related to each of these concepts. Here we present previous work on computational tools for psychology, domain specific and solver aided languages, and on working in combinatorial search spaces.

\paragraph*{Psychology Tools, Experimental Design \& Replicability}

There are many computational tools used by psychologists, however, they automate stimulus presentation for running experiments, not experimental design. Some popular toolboxes include PsyScope \cite{cohen1993psyscope} (now called E-Prime), PsychoPy \cite{mathot2012opensesame}, OpenSesame \cite{peirce2009generating}, Presentation and PsychToolBox. PsyNeuLink is a tool for running computational models on experimental tasks. All of these tools are helpful for automating psychology experiments, however they all fill a different role than SweetPea: ideally SweetPea would easily interface with these tools to pipeline experiments from their design to their execution.

SweetPea addresses issues arising from the replicability crisis in psychology, by creating reproducible and sharable experimental sequences. The replicability crisis has been well documented, espcially in recent years-- see  \cite{pashler2012editors}, \cite{pashler2012replicability}, \cite{simons2014value}, \cite{schmidt2016crisis},  \cite{maxwell2015psychology}, \cite{stroebe2016most}.
In the future, we hope that SweetPea can provide an even higher-level interface where a psychologist can specify a contrast they wish to study, and SweetPea can choose the best constraints to impose on the design. Related work has been done on optimal experiment design by \cite{myung2009optimal}.


\paragraph*{Solver Aided Languages}

SweetPea can be thought of as a domain-specific interface to any tool which solves boolean formulas, and specifically to the SAT sampler Unigen, which provides the near-uniformity guarantee. There are other \emph{solver aided languages} such as, Dafny \cite{leino2013developing}, which is the language for interacting with the solver z3 in, and Sketch \cite{bodik2017domain} which "Domain-Specific Symbolic Compilation". There is also Rosette \cite{torlak2014lightweight} which is a platform for developing solver aided DSLs. Finally, Hyperkernel \cite{nelson2017hyperkernel} discusses the process of co-designing a system to be amenable to verification, analogously to how the SweetPea language is designed to be amenable to translation into SAT.


\paragraph*{Combinatorial Search Spaces}
?

\cite{meel2016constrained}

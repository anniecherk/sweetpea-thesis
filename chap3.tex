%%% -*-LaTeX-*-

\chapter{Related Work}

This is a chapter.


\section{Psychology Toolboxes}

- psyScope
- psychoPy
- OpenSesame

\subsection{Reproducibility Crisis}

\blah

\section{Domain Specific Languages}

- this is a huge area, not sure need to cite anything

\subsection{Solver Aided Languages}

- rosette
- sketch: "Domain-Specific Symbolic Compilation"
- dafny maybe

- hyperkernel: co-designing a language and the verification


\section{Combinatorial Search Spaces}

- finding solutions in a large search space-- no really, very large
- how large?
- so large
- what is the nature of our search constraints? things like 60 red words, 60 blue words (in Stroop, see chapter 2).


\subsection{Sampling Methods}

- sampling is the problem of finding solutions
- could try solutions at random: turns out they are sparse (most examples don't have 60 red, 60 blue)
- could try to generate all solutions: turns out there are too many (lots of possible arrangements)

- could use MCMC, but doesn't provide guarantees
- this project is *really* about providing this guarantee that we're not introducing bias because this is a huge deal

\subsection{Boolean Satisfiability}

- SAT is a classic problem, NP-complete
- SAT solvers happen to be really good
- To specify something in SAT, you use variables and specify invariants
- the SAT solver finds an assignment that satisfies the invariants
- we use a SAT sampler which finds multiple assignments, and with guarnatees

- an alternative is using SMT constraints
- we compile to SAT because unigen is available; to use a different tool we could pretty easily swap out the backend


\subsection{Uniform Sampling}

- re-read unigen paper
- what problem is it solving? want uniform for coverage
- who else cares about this problem
- how is it solved: universal hash functions
- what alternatives exist

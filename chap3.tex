%%% -*-LaTeX-*-


\chapter{Related Work}

SweetPea is a computation tool for pyschologists; it is a solver-aided language which translates between the vocabulary which psychologists employ and a SAT formula. To the best of our knowledge, SweetPea is the only system to address this specific problem, however there is a large body of work related to each of these concepts. Here we present previous work on computational tools for psychology, domain specific and solver aided languages, and on working in combinatorial search spaces.

\paragraph*{Psychology Tools, Experimental Design \& Replicability}

There are many computational tools used by psychologists, however, they automate stimulus presentation for running experiments, not experimental design. Some popular toolboxes include PsyScope \cite{cohen1993psyscope} (now called E-Prime), PsychoPy \cite{mathot2012opensesame}, OpenSesame \cite{peirce2009generating}, Presentation and PsychToolBox. PsyNeuLink is a tool for running computational models on experimental tasks. All of these tools are helpful for automating psychology experiments, however they all fill a different role than SweetPea: ideally SweetPea would easily interface with these tools to pipeline experiments from their design to their execution.

Optimal experiment design \cite{myung2009optimal}

Editors’ introduction to the special section on replicability in psychological science: A crisis of confidence? \cite{pashler2012editors}

Is the replicability crisis overblown? Three arguments examined \cite{pashler2012replicability}
The value of direct replication \cite{simons2014value}

The crisis of confidence in research findings in psychology: Is lack of replication the real problem? Or is it something else? \cite{schmidt2016crisis}

Is psychology suffering from a replication crisis? What does “failure to replicate” really mean? \cite{maxwell2015psychology}

Are most published social psychological findings false? \cite{stroebe2016most}


\paragraph*{Solver Aided Languages}

SweetPea is an interface to the SAT sampler Unigen. The benefit of Unigen is it provides strong statistical guarantees, but the drawback is that it only understands boolean formulas. To leverge the power of this tool

- rosette \cite{torlak2014lightweight}

- sketch: "Domain-Specific Symbolic Compilation" \cite{bodik2017domain}

- dafny (the z3 language) \cite{leino2013developing}

- hyperkernel: co-designing a language and the verification \cite{nelson2017hyperkernel}


\paragraph*{Combinatorial Search Spaces}
?


\paragraph*{Uniform Sampling}

\cite{meel2016constrained}

- what problem is it solving? want uniform for coverage

- who else cares about this problem

- how is it solved: universal hash functions

- what alternatives exist
